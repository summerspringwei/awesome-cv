%-------------------------------------------------------------------------------
%	SECTION TITLE
%-------------------------------------------------------------------------------
\cvsection{Research}


%-------------------------------------------------------------------------------
%	CONTENT
%-------------------------------------------------------------------------------
\begin{cventries}

%---------------------------------------------------------
  \cventry
    {优化难:自动子图映射与调度} % Job title
    {异构融合的移动端AI推理加速} % Organization
    {北京} % Location
    {2019.04-至今} % Date(s)
    {
      \begin{cvitems} % Description(s) of tasks/responsibilities
        \item {问题:移动端集成多种异构加速芯片,如何充分利用异构芯片在功耗敏感的情况下加速AI推理}
        \item {难点:(1)移动设备巨大性能差异(2)日趋复杂的DNN负载(3)轻量级异构融合的并行执行}
        \item {设计:(1)基于权值的子图划分(2)基于线性规划和贪心的算子分配与调度(3)Cache-aware的优化和轻量级调度}
        \item {结果:典型DNN模型(Inception-V3/V4,Pnasnet-large)等相比State-of-Art最多$1.65\times$加速($2$计算单元)}
        \item {专利一篇:在异构处理单元上执行深度神经网络的方法,申请号:202010493830.8}
        \item {\textbf{{\color{awesome-red}论文1}:《HOPE: A Heterogeneity-Oriented Parallel Execution Engine for Inference on Mobiles》投稿至CGO21(CCF-B)}}
        \item {\textbf{{\color{awesome-red}论文2}:《On Re-targeting the AI Programming Framework to New Hardwares》,NPC18(CCF-C,第四作者)}}
      \end{cvitems}
    }

%---------------------------------------------------------
  \cventry
    {编程难:自动选择最优配置} % Job title
    {面向端云协同AI推理的DNN调优框架} % Organization
    {北京} % Location
    {2017.06 - 2019.04} % Date(s)
    {
      \begin{cvitems} % Description(s) of tasks/responsibilities
        \item {问题:DNN在移动端和云端协同的推理成为DNN部署新方法}
        \item {难点:(1)如何准确的测量延迟和功耗(2) 如何切分DNN负载最小化延迟和功耗(3)如何选择最优的软硬件配置}
        \item {设计:(1)动态的layer-wise profiling机制(2)自动调优框架:DNNTune自动发现最优切分点和最佳软硬件配置}
        \item {结果:(1)端云协同最多$1.66\times$加速比、节省$15\%$能耗}
        \item {结果:(2)刻画各种DNN(CNN/LSTM/MLP)在各种平台(高中低、CPU/GPU/AI、手机/嵌入式)各种配置的性能}
        \item {\textbf{{\color{awesome-red}论文3}:《DNNTune: Automatic Benchmarking DNN Models for Mobile-cloud Computing》{\color{awesome-red}ACM-TACO (CCF-B),HiPEAC会议报告}}}
        \item {\textbf{\color{awesome-red}DNNTune成为过去一年ACM TACO下载量最高的论文(截止目前)。}}
        \item {\textbf{{\color{awesome-red}论文4}:《Characterizing DNN Models for Edge-Cloud Computing》(poster),发表在\textbf{IISWC-2018(Benchmark旗舰会议)}}}
      \end{cvitems}
    }

%---------------------------------------------------------
  \cventry
    {CIKM-2015 Competition Top-20} % Job title
    {Web查询的分类算法} % Organization
    {天津} % Location
    {2014.10 - 2015.10} % Date(s)
    {
      \begin{cvitems} % Description(s) of tasks/responsibilities
        \item {\textbf{{\color{awesome-red}论文5}:基于百度搜索引擎数据集的图模型的Web查询分类算法,《Graph-Based Web Query Classification》WISA-2015(EI检索)}}
      \end{cvitems}
    }

%---------------------------------------------------------

\end{cventries}
