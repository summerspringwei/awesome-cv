%-------------------------------------------------------------------------------
%	SECTION TITLE
%-------------------------------------------------------------------------------
\cvsection{Projects}


%-------------------------------------------------------------------------------
%	CONTENT
%-------------------------------------------------------------------------------
\begin{cventries}

%---------------------------------------------------------
  \cventry
    {主要参与人员,负责DNN模型在异构融合加速器的算子映射和任务调度} % Role
    {面向异构融合数据流加速器的编程模型及编译器优化-国家重点研发计划子课题} % Event
    {北京} % Location
    {2017.12 - 2019.12} % Date(s)
    {
      % \begin{cvitems} % Description(s)
      %   \item {}
      %   \item {}
      % \end{cvitems}
    }

%---------------------------------------------------------
  \cventry
    {主要参与人员,负责将原本在CPU上使用MKL加速的FFT应用,迁移到NVIDIA Jetson嵌入式开发板,验证ARM CPU+ NVIDIA GPU的异构加速效果} % Role
    {张量计算异构处理框架验证项目} % Event
    {北京} % Location
    {2018.04-2018.06} % Date(s)
    {
    %   \begin{cvitems} % Description(s)
    %     \item {Introduced CTF(Capture the Flag) hacking competition and advanced techniques and strategy for CTF}
    %   \end{cvitems}
    }

%---------------------------------------------------------
  \cventry
    {主要参与人员,通过在Spark框架中代码插桩,来测量RCA应用在Spark上各个阶段的延迟} % Role
    {5G RCA应用在大数据处理框架的评估建模} % Event
    {北京} % Location
    {2016.02-2016.08} % Date(s)
    {
      % \begin{cvitems} % Description(s)
      %   \item {Introduced basic procedure for penetration testing and how to use Metasploit}
      % \end{cvitems}
    }

%---------------------------------------------------------
%---------------------------------------------------------
\cventry
{项目负责人,基于天津市真实的出租车交通流数据,预测在短时间内的道路车辆运行速度与拥堵状况} % Role
{国家大学生创新创业训练项目(国家级):基于信息融合的短时交通流速度预测} % Event
{天津} % Location
{2014.05-2015.05} % Date(s)
{
  % \begin{cvitems} % Description(s)
  %   \item {Introduced basic procedure for penetration testing and how to use Metasploit}
  % \end{cvitems}
}

\end{cventries}
